			
%%%%%%%%%%%%% 7 %%%%%%%%%%%%%	
\chapter{Conclusion}
%%%%%%%%%%%%%%%%%%%%%%%%%%%%%

\section{Summary of Results}
	My analysis of the ANITA-III detected 20 UHE CR EAS events over the 22 day flight over Antarctica.  It also observed an inverted polarity CR signal that is consistent of an upward going $\tau$ decay from a UHE $\nu_{\tau}$ transiting through the earth.  Current models for the cross sections of these astrophysical particles at UHE are not in agreement with this detection.  Further study of the event and the systematic uncertainties presented in this analysis must be undertaken before any conclusions can be drawn.


\section{Future Analysis Work on the ANITA-III payload}
	During my analysis I discovered several calibration issues which needed to be addressed.  This includes the off axis response and cross-pol leakage of the antennas, which is not well constrained.  Additionally, an absence of 45$^\circ$ polarized calibration signals presents challenges for accurage alignment of the phase centers for the two different polarization ridges.  These uncertainties have a strong effect on energy reconstruction and plane of polarization angle measurement for the detected signals, and will need to be corrected for an accurate result.  The geomagnetic angle is of particular interest, as it provides an additional parameter with which to discriminate between background and signal events.


\section{ANITA-IV Flight and Improvements}
	The ANITA-IV payload, an upgraded version of the ANITA-III instrument, flew in the 2016-2017 Antarctic balloon campaign.  It corrected several issues that improve exposure, sensitivity, and provide additional constraints on measurement uncertainties that plagued ANITA-III.  These include a tunable notch filter on each channel to eliminate satellite background, and additional calibration pulser measurements.
	
	The results of this analysis significantly enhance the value of the recovered ANITA-IV flight data.  With increased exposure and sensitivity, additional inverted polarity detections would confirm the event found in this analysis.
	
	The future looks bright for ANITA.
	
	
	%crab