
%%%%%%%%%%%% 1 %%%%%%%%%%%
\chapter{Introduction and Motivation}
		The universe is vast, surprising, and has just begun to be explored by humanity.  As telescopes become larger and more sensitive, we as a species can peer further away from our tiny blue speck with increasing clarity.  However, the absorption of high energy photons in the interstellar medium from fundamental physics interactions clouds much of the energetic universe from our gaze.  Nevertheless, there is a constant flux of hadronic particles incident on earth from various cosmic accelerators that do not suffer as severely from attenuation over universe scale distances.  These messenger particles provide a new window into regions of the universe that were previously inaccessible.
%%%%%%%%%%%%%%%%%%%%%%%
\section{Cosmological Theory}
	\subsection{Cosmic Ray Detection History}
	Cosmic rays (CRs) were first observed over a century ago, when an unexpected increase in ionizing radiation was observed as altitude increased (Figure \ref{fig:HessKol}).  Previously, it was expected that terrestrially measured radiation was caused by radioactive decays within the earth's crust, however this additional radiative component was indicative of a source far from the Earth.\cite{HessCosmicRay}  This discovery spawned a search for the sources of these particles raining down from the heavens.  In the subsequent hundred years, hundreds of experiments have taken up the task of measuring and characterizing cosmic rays incident on the atmosphere, utilizing a variety of techniques.\cite{Olive:2016xmw}  The search for cosmic accelerators able to produce the highest energy cosmic rays that have been experimentally measured remains an unsolved problem in physics that persists to today. 
	
\begin{figure}
\centering
	\includegraphics[width=0.7\textwidth]{figures/HessKol}
	\caption{Measurements of increasing ionizing radiation as a function of energy, by Hess in 1912 (left) and Kolhörster in 1913 (right), which provided strong evidence for an extra-terrestrial particle flux
	\cite{HessKolPic} }
\label{fig:HessKol}
\end{figure}
	
	\subsection{Cosmic Ray Physics and Cosmology}
	While the sources of cosmic rays remain unknown, their nature opens up the doors to many novel measurements of cosmological phenomenon that occur at extreme distances and energies.  Much of the universe is opaque to the traditional astronomical messenger particle, the photon, at high energies.  High energy gamma rays undergo electron-positron pair production in the presence of a magnetic field, effectively halting their travel before they can reach earth.(Figure \ref{fig:observableUniverse})\cite{RevModPhys.41.581}.  Additionally, the energies of some particles detected by cosmic ray observatories, on the order of $10^20$eV, dwarf the energies produced at terrestrial particle accelerators.  These particles can shed light on the world in the ultra high energy (UHE) regime.

\begin{figure}
\centering
	\includegraphics[width=\textwidth]{figures/ObservableUniverse}
	\caption{Interaction distances for photons and protons.  Shaded regions represent regions of the universe which are opaque to an astronomical particle. Credit to Dr. Peter Gorham for this plot.}
\label{fig:observableUniverse}
\end{figure}
	
	The observed energy spectrum of these hadronic particles introduces additional puzzles. As energy increases, the number of candidate sources for cosmic rays diminishes, leaving few to no intra-galactic candidates.\cite{RevModPhys.71.S33}  Flux density as a function of energy, shown in Figure \ref{fig:cosmicrayflux} is well measured up to the EeV energy scale with modern experiments, however in the UHE region, those particles with energies above $10^{18}$eV, statistical limitations caused by the incredibly low flux, on the order of 1/km$^2$/century, prevent accurate pointing to source locations.  Additional measurements are required to determine the transition between CRs emanating from galactic and extra-galactic sources.

	Using cosmic rays as a astronomical messenger particle presents new difficulties as well.  At lower energies, charged particles such as cosmic rays will have their courses altered by the Lorentz force while transiting the magnetic fields of stellar objects. Subsequently, inverting their incoming detection vector will no longer yield accurate pointing back to the source accelerator.  This is depicted in Figure \ref{fig:CosmicRayDeviation}.  Higher energy cosmic rays are proportionally less effected by magnetic fields due their increased rigidity, however, once more, the extremely low event statistics make it impossible to point to objects with any level of certainty. 
	
	The composition of cosmic rays at the highest energies remains under investigation as well.  An extensive air shower produced from a CR interacting with an atmospheric molecule has identical secondary detection characteristics whether the initial source particle was heavy ions or protons.  Recent measurements studying the maximum shower depth of air showers suggests a mixed composition, and possibly a hint at a transition from galactic to extragalactic source accelerator at EeV energies.\cite{AugerXmax}\cite{AugerXmaxDiscussion} Additional experimental observations of these cosmological particles will yield a better understanding of both the structure of matter and energy within the universe at large, as well as fundamental physics processes energetically unachievable from existing collider experiments.  The creation and propagation of ultra high energy cosmic rays through the universe opens a window to understanding of extreme high energy physics.

		
\noindent
\begin{figure}
\centering
	\includegraphics[width=\textwidth]{figures/AugerXmax}
	\caption{Measurements of cosmic ray maximum shower depth by the Auger observatory.  Modeled lines show theoretical predictions for two cosmic ray compositions, iron and protons\cite{AugerXmax}}
	\label{fig:AugerXmax}
\end{figure}

\noindent		
\begin{figure}
	\includegraphics[width=\textwidth]{figures/CosmicRayDeflection}
	\caption{Top: A simplified diagram of lorentz-force induced curvature in low energy cosmic rays.  Bottom: An example of why magnetic deflection distorts the observed CR source location}
	\label{fig:CosmicRayDeviation}
\end{figure}
		

\noindent		
%\begin{wrapfigure}{L}{0.5\textwidth}
\begin{figure}
	\includegraphics[width=\textwidth]{figures/CosmicRayFluxMeasurements}
	\caption{The all-particle spectrum as a function of energy-per-nucleus from measurements\cite{Olive:2016xmw}}
	\label{fig:cosmicrayflux}
\end{figure}
%\end{wrapfigure}		
		


\section{Neutrino astrophysics and the GZK interaction}
		The steep decline in observed flux measured by detectors alludes to an interaction mechanism that opens up a new detection prospect for cosmic rays.  Dubbed, "the ankle" in Figure \ref{fig:cosmicrayflux}, the sharp decrease at 10$^{19}$eV corresponds to an interaction of a rest proton with a high energy gamma ray through a delta resonance.  Shifting the frame of reference to a relativistic proton colliding with a low energy photon, one arrives at an accurate description of a cosmic ray transiting the cosmic microwave background (CMB) radiation that isotropically pervades the universe.\cite{WMAPCMBResults}  This interaction was predicted by Greisen–Zatsepin–Kuzmin, and represents a theoretical high energy limit on particles coming from outside of the galaxy, known as the GZK limit, at 5x10$^{19}$ eV, above which particles will scatter off the CMB and decline in energy.\cite{GZK}  Observatories measuring UHE cosmic rays have detected a decrease in the quantity of observed particles consistant with a GZK induced suppression.\cite{GZKMeasurement} Cosmic ray particles were also observed exceeding this limit however, suggesting an intra-galactic source of unknown origin.  The low statistics of particles observed at this energy prevent identification of a specific source.  Since particles are present at these high energies, it would also be expected that other regions of the universe also contain accelerators capable of creating cosmic rays in excess of the GZK limit, thus motivating a UHE neutrino flux.
		
		The GZK process has multiple channels, each of which produce an ultra high energy neutrino (UHE$\nu$) messenger particle that could subsequently be detected on earth (Figure \ref{fig:GZKDiagram})\cite{GZK}.  Though the flavor ratio produced in a GZK interaction is not equally distributed, flavor oscillation during their long time-scale traverse to earth will result in an observable flavor ration of 1:1:1.  The cross section of neutrinos has been measured in accelerator facilities to be vanishingly small at modern accelerator scale energies.\cite{neutrinoCrossSectionMeasurements}  However, at energies out of reach of modern accelerators, cosmic accelerators could illuminate our understanding of neutrino cross section.  The neutrino cross section at energies above those measured by accelerator facilities has been estimated by using standard model particle physics.\cite{neutrinoCrossSectionExtrapolation} This small cross section allows the GZK interaction messenger neutrino to traverse unimpeded through the interstellar medium before subsequently interacting with earth.  Additionally, the uncharged nature of neutrinos allow them to travel in a straight line from their source locations without Lorentz deflection from intervening magnetic fields.  Cosmologically propagating neutrinos have an extremely small cross section, such that their path length through empty space is essentially infinite, allowing an unparalleled view into the depths of the cosmos.

\noindent		
%\begin{wrapfigure}{L}{0.5\textwidth}
\begin{figure}
	\includegraphics[width=\textwidth]{figures/GZKDiagram}
	\caption{A diagram of the possible resultant particles of a GZK interaction between a photon and the CMB.}
	\label{fig:GZKDiagram}
\end{figure}
%\end{wrapfigure}		
			
		
	\subsection{Unexplained UHE Source Mystery}
		Though there are numerous cosmic ray sources that have strong theoretical motivation, above 10$^{17.5}$eV the number of candidates becomes constrained and an extra-galactic source hypothesis becomes required. \cite{RevModPhys.71.S33}  Due to the higher flux of nearby low energy cosmic ray sources, many experiments have gathered evidence to support source candidates within our galaxy as accelerators.  However, above the GZK suppression limit there remains little collected experimental evidence that describes the transition to extra-galactic sources. These higher energy objects include such objects and Active Galactic Nuclei (AGN), supernova, quasars, gamma-ray bursts, as seen in Figure \ref{fig:cosmicAccels}  However, the small statistics afforded by incident particles at the EeV end of the spectrum makes identifying their acceleration mechanisms much more difficult, and requiring extremely large detector volumes.

\begin{figure}
	\centering
	\includegraphics[width=\textwidth]{figures/cosmicAccelerators}
	\caption{Theorized cosmic accelerators plotted by their size and peak magnetic field.  The dashed line denotes 
a field strength capable of accelerating protons to $10^{18}$ eV. \cite{RevModPhys.71.S33} }
	\label{fig:cosmicAccels}
\end{figure}


\section{Neutrino Detection on Earth}
	The neutrinos produced in GZK interactions are messenger particles that carry information about UHECRs produced outside the galaxy to earth at energies above the GZK suppression effect.  Astronomical neutrino telescope observations have been carried out at lower energies nearly since the discovery of the particle itself. Recently however, there has been a notable increase in detector scale and energy sensitivity.  Since the term "cosmic ray" describes any high energy particle incident on earth (including gamma rays), for the purposes of this dissertation, neutrinos fit the description of cosmic rays.
	
	The largest notable difference between observations of UHE neutrinos and cosmic rays is the medium in which they are most likely to interact.  While cosmic rays have large cross sections and are predicted to interact with even low density media, such as the atmosphere, neutrinos require a medium with far higher density.  Interestingly, both particle interaction lengths can be described with a traversed density unit, or specifically for the case of cosmic rays the atmospheric depth, $\frac{g}{cm^{2}}$.  For UHE$\nu$s however, this term becomes larger than the integrated total density of the atmosphere regardless of incoming slant angle.  The neutrino will therefore either skim the atmosphere entirely if shallow enough, or interact within the earth if too steep.  However, if a dense dielectric solid is introduced or utilized at the Earth's surface by an opportunistic experimenter, it is possible to capture neutrino interactions within the field of view of detection equiptment.  This equiptment can capture the induced electromagnetic showers from a high energy particle shower.
	
	\subsection{Charged current and Neutral Current interactions} 
		Neutrinos have two distinct interaction classifications, neutral current and charged current.  A neutral current interaction occurs through the exchange of a $Z^{0}$ particle, and transfers some of its energy and momentum to the particle it interacted with.  A charged current interaction involves the exchange of a $W^{\pm}$, and results in the neutrino being converted into the a similarly flavored and charged lepton.  The two interactions have differing cross sections and have different characteristic signals.
				

	\subsection{Extensive Air/Ice Showers}
		A highly energetic particle interacting with the hadronic matter present in Earth's atmosphere, or in any  dense material, creates an extended series of particle production, scatterings, and decays.  This extensive shower of particles scattered and created from a single extra-terrestrial high energy source particle has been appropriately dubbed an Extensive Air Shower (EAS).  These showers produce both traveling messenger particles that can be observed through weakly interacting secondary scatterings in instrumented transparent scintillating media, as well as through primary radiation, which is the topic of this thesis.  This primary radiation is stimulated by the appearance of moving charged particle pairs that travel along the principle shower axis.  There are two main electromagnetic radiative effects, detailed below and depicted in Figure \ref{fig:EASRadiation}, that can be observed in the  VHF (3MHz to 300MHz) and UHF (300MHz to 3GHz) bands of the electromagnetic spectrum.  This region of the spectrum has very good transmissive properties in the atmosphere, as well as within many dense solids.\cite{Besson2009348}\cite{VuFind-000215473}\cite{Barrella:2010vs}
		

\begin{figure}
	\centering
	\includegraphics[width=\textwidth]{figures/EASRadiation}
	\caption{Simplified schematic of two EAS radiative trasmission mechanisms, geomagnetic (left) and charge-excess, or Askaryan, radiation (right)}
	\label{fig:EASRadiation}
\end{figure}

	
	\subsection{Geomagnetic Radiation}
		The primary and strongest radiative effect in an EAS is from a longitudinal charge separation created from a Lorentz force as the newly created grouping of charged particles move through the geomagnetic field of the Earth.  As the shower is required to have zero net charge (A CR will have a total incident charge of +1 and a neutrino has none), an equal number of positively and negatively charged particles is expected to be generated.  The Lorentz force, $F=q(\vec{E}+\vec{v}\times\vec{B})$, acting on oppositely charged particles will move them in opposite directions, inducing a spacial separation.  This charge separation emits radiation with a linear polarization orthogonal to the magnetic field and the shower axis with an intensity proportional to the incoming shower energy and magnetic field.
	\subsection{Askaryan Effect}
		A secondary radiation component, theorized by Gurgen Askaryan in 1962, as a build up of negative charge at the front of the shower core as electrons are created and scattered forward, while their positron pairs are annihilated.\cite{Askaryan:1962hbi}  This creates a charge separation between the negatively charged shower front and the positively charged shower path.  This relativistic velocity charge buildup will radiate coherently at critical off angles where their created wavelengths constructively interfere.  This is visible as a sharp broad spectrum impulse, as a stationary viewer observes a enormous charge flux.\cite{PhysRevD.84.103003}  This effect has much more recently been measured at particle accelerators in a variety of materials, including ice,\cite{PhysRevLett.99.171101} salt\cite{PhysRevD.72.023002}, and silica sand\cite{PhysRevLett.86.2802}.  An example accelerator initiated Askaryan pulse in ice measured by the ANITA electronics is visible in Figure \ref{fig:ANITASLACPulse}
		
		
\begin{figure}
\centering
	\includegraphics[width=\textwidth]{figures/AskaryanSimulation}
	\caption{The electric field of an Askaryan pulse observed close to the critical angle in the far field generated by a mathematical model(red), and with the ZHS simulation package(blue) \cite{PhysRevD.84.103003} }
	\label{fig:AskaryanSimulation}
\end{figure}

\begin{figure}
\centering
	\includegraphics[width=\textwidth]{figures/ANITASLACImpulse}
	\caption{An example accelerator initiated Askaryan pulse in ice measured by the ANITA electronics  The dispersion in the signal is introduced by the band-pass filtering in the RF signal chain\cite{PhysRevLett.99.171101} }
	\label{fig:ANITASLACPulse}
\end{figure}


	\subsection{Cherenkov Radiation}
		Considering a realistic index of refraction for the dielectric medium in which a particle shower occurs results in a boost to both primary sources of shower radiation at distances slightly off the shower axis.  This is caused by constructive interference induced by the relativistically propagating shower through a dielectric, in which light propagates at $<c$, at a critical off angle, which corresponds with Cherenkov radiation.  This process is shown in Figure \ref{fig:Cherenkov}.
		

\begin{figure}
\centering
	\includegraphics[width=0.5\textwidth]{figures/Cherenkov}
	\caption{An example diagram of the Cherenkov critical angle where the radiative components of a CR shower experience a signal boost.}
	\label{fig:Cherenkov}
\end{figure}


	\subsection{Tau-$\nu$ Specific Detection Prospects}
		Though the Earth becomes opaque to neutrinos at high energies, secondarily created particles such as a Tau lepton from a similarly flavored neutrino undergoing a charged-current interaction have a regeneration effect that opens a larger field of view than just the sliver of ice at the horizons.  These Taus, which have a relatively short half-life, will then decay in the atmosphere, initiating an EAS that travels upwards from the surface of the Earth.  A tau neutrino thus has a higher accepted incoming angle, as it can traverse further through the dense rock of the Earth without being absorbed fully.  A UHE$\nu_{\tau}$ will behave characteristically similar to a UHECR, except that it will have a flipped polarity to a reflected down-going EAS, and a matched polarity to a directly viewed EAS.  A search for UHE$\nu_{\tau}$ particles was done for the ANITA1 data set, and furthering the search for the ANITA3 flight is a primary motivator of this thesis.  This "double-bang" tau neutrino interaction has been simulated, and other experiments are searching for their signals\cite{PhysRevD.86.022005}. 

\section{ANITA}
	Up to this point, I have mainly motivated a detectable physics hypothesis from established theories and measurements.  From here, I can introduce a detection platform for these physics phenomenon, the ANtarctic Impulsive Transient Antenna (ANITA).  EASs present in the atmosphere or in the dense, radio transparent, dielectric solid ice sheet covering the souther continent of Antarctica will produce unique and characteristic signals through the above particle interactions and radiative effects.  A high altitude, long duration, telescope platform would have an extremely large field of view, thus increasing the possibility of observing one of these rate interactions.  Additionally, downwardly moving atmospheric CR shower events can be both observed by the payload directly, or as reflections off the ice that covers the Antarctic continent.  These reflected pulses will have an inverted phase from the theoretical models of an EAS.  The following chapters detail the instrument, analysis, and simulation of the third ANITA flight.




