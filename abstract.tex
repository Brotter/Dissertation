%%%%%%%%%%%%%%%%%%%%%%%%%%%%%% -*- Mode: Latex -*- %%%%%%%%%%%%%%%%%%%%%%%%%%%%
%% uhtest-abstract.tex -- 
%% Author          : Ben Rotter
%% Created On      : Fri Oct  2 16:30:18 1998
%% Last Modified By: Ben Rotter
%% Last Modified On: Fri Oct  2 16:30:25 1998
%% RCS: $Id: uhtest-abstract.tex,v 1.1 1998/10/06 02:06:30 rbrewer Exp $
%%%%%%%%%%%%%%%%%%%%%%%%%%%%%%%%%%%%%%%%%%%%%%%%%%%%%%%%%%%%%%%%%%%%%%%%%%%%%%%
%%   Copyright (C) 1998 Robert Brewer
%%%%%%%%%%%%%%%%%%%%%%%%%%%%%%%%%%%%%%%%%%%%%%%%%%%%%%%%%%%%%%%%%%%%%%%%%%%%%%%
%% 

\begin{abstract}
%Basically here I want to talk about the ANITA3 instrument, the theory that is it stuyding (Cosmogenic neutrinos and cosmic rays), and the results of the recent flight.  I should really mention the link between the cosmic rays (even though it seems like the energies we see are lower than the knee cutoff region) and maybe even add in something about darkmatter strangelets in an appendix or something.  I should probably be doing something to link those two things together


%% This is Ben Rotter, and I'm going to write a dissertation.
This dissertation is an overview of one experimental study of the highest energy parrthyjuikticles created in the cosmos, their propogation through space, and their interaction with other media.  The story begins with the discovery of ultra-high energy (UHE) cosmic rays, those with energies on the order of EeV.  Particles of this energy have been observed in multiple experiments interacting with matter in the atmosphere and subsequently creating large, energetic particle showers which can be detected by space and ground based detectors.  Due to the interesting nature of rare, extremely high energy particles with unknown source, much study was poured into their study.  From this, precision measurements of the flux density over a wide range of energies have been taken, and several interesting features are found within the spectrum.  Most notibly, at the highest observed energy edge of the spectrum, a sudden decrease in flux (or knee) can be observed.  This energy happens to be situated at a resonance for an interaction between a relativistic particle and the low energy photon background radiation, the CMB.  This interaction involves the creation of a delta resonance and results in the creation of a neutrino (neutral current and charged current).  This highly reletivistic neutrino should make up the resulting lack of flux in the cosmic ray spectrum, and has yet to be observed.  The discovery of this cosminogenic neutrino is the main scope of the ANtarctic Impulsive Transient Antenna (ANITA) telescope payload.

Due to their low flux density, observation of UHE particles is dependent on secondary interations that emit characteristic messenger particles.  These secondary particles increase the observable cross section to a point where it can be realistically measured by a detector with a sufficiently instrumented volume.  In the case of ANITA, the messenger particle is coherant radio frequency emmision, which has a beneficial attenuation length in both air and ice.  As power is only minimally dissipated as a function of distance from interaction site, the total instrumented volume is proportional to the surface area of the observable ice, which is a function of the altitude of the payload.

\end{abstract}
