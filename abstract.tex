%%%%%%%%%%%%%%%%%%%%%%%%%%%%%% -*- Mode: Latex -*- %%%%%%%%%%%%%%%%%%%%%%%%%%%%
%% uhtest-abstract.tex -- 
%% Author          : Ben Rotter
%% Created On      : Fri Oct  2 16:30:18 1998
%% Last Modified By: Ben Rotter
%% Last Modified On: Fri Oct  2 16:30:25 1998
%% RCS: $Id: uhtest-abstract.tex,v 1.1 1998/10/06 02:06:30 rbrewer Exp $
%%%%%%%%%%%%%%%%%%%%%%%%%%%%%%%%%%%%%%%%%%%%%%%%%%%%%%%%%%%%%%%%%%%%%%%%%%%%%%%
%%   Copyright (C) 1998 Robert Brewer
%%%%%%%%%%%%%%%%%%%%%%%%%%%%%%%%%%%%%%%%%%%%%%%%%%%%%%%%%%%%%%%%%%%%%%%%%%%%%%%
%% 

\begin{abstract}

%% Lets try this again
Ultra-high energy cosmic rays (UHECRs), those with energies above $10^{18}$~eV, have been observed colliding with matter in the Earth's atmosphere by a variety of experiments since the 1960's. These particles are among the highest observed locally on earth, with energies many orders of magnitude higher than those produced in terrestrial accelerators.  Due to the extreme energies of these rare particles, much study has been poured into determining their cosmic source accelerators, a search which continues today.  This dissertation explores the third flight of the ANtarctic Impulsive Transient Antenna (ANITA) telescope payload in the 2014-2015 Antarctic season, which has a unique opportunity for novel observations of UHE particles.  It covers the cosmological theory, payload overview, detector calibration, signal simulation, and presents results of a search for UHECRs and UHE$\nu_{\tau}$s with the third ANITA flight.

\end{abstract}















%Basically here I want to talk about the ANITA3 instrument, the theory that is it stuyding (Cosmogenic neutrinos and cosmic rays), and the results of the recent flight.  I should really mention the link between the cosmic rays (even though it seems like the energies we see are lower than the knee cutoff region) and maybe even add in something about darkmatter strangelets in an appendix or something.  I should probably be doing something to link those two things together


%% This is Ben Rotter, and I'm going to write a dissertation.
%This dissertation is an overview of one experimental study of the highest energy particles created in the cosmos, their propogation through space, and their interaction with other media and subsequent detection on Earth. Ultra-high energy cosmic rays (UHECRs), those with energies on the order of $10^{18}eV$, have been observed colliding with matter in the Earth's atmosphere.  These interactions result in the creation of large, energetic particle showers which can be detected by space and ground based detectors.  Due to the extreme energies of these rare particles, much study was poured into determining their cosmic source accelerator.  From this, precision measurements of the flux density over a wide range of energies have been taken, and several interesting features are found within the spectrum.  Most notibly, at the highest observed energy edge of the spectrum, a sudden decrease in flux (or knee) can be observed.  This energy happens to be situated at a resonance for an interaction between a relativistic particle and the low energy photon background radiation, the CMB.  This interaction involves the creation of a delta resonance whose subsequent decay includes a propagating neutrino.  This highly reletivistic neutrino is hypothesized to have high enough flux to be detectable, and has yet to be observed.  The discovery of this cosminogenic neutrino is the main scope of the ANtarctic Impulsive Transient Antenna (ANITA) telescope payload.  This thesis details the theory, experimental design, and analysis of the third flight of the ANITA experiment, culminating in a defendable limit on the flux density of these mysterious particles, as well as a search for UHECRs.
