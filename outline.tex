%%%%%%%%%%%%%%%%%%%%%%%%%%%%%% -*- Mode: Latex -*- %%%%%%%%%%%%%%%%%%%%%%%%%%%%
%% uhtest-body.tex -- 
%% Author          : Robert Brewer
%% Created On      : Fri Oct  2 16:30:37 1998
%% Last Modified By: Robert Brewer
%% Last Modified On: Mon Oct  5 16:01:29 1998
%% RCS: $Id: uhtest-body.tex,v 1.1 1998/10/06 02:07:14 rbrewer Exp $
%%%%%%%%%%%%%%%%%%%%%%%%%%%%%%%%%%%%%%%%%%%%%%%%%%%%%%%%%%%%%%%%%%%%%%%%%%%%%%%
%%   Copyright (C) 1998 Robert Brewer
%%%%%%%%%%%%%%%%%%%%%%%%%%%%%%%%%%%%%%%%%%%%%%%%%%%%%%%%%%%%%%%%%%%%%%%%%%%%%%%
%% 
%!!!!!!!!!!!!!!!!!!!!!!!!!!!!!!!!!!!!!!!!!!!!!!!!!!!!!!!!!!!!!!!!!!!!!!!!!!!!!!
%!NOTE: This example file has been prepared according to the University of
%!      Hawaii Style & Policy Manual for Theses and Dissertations dated
%!      "Revised September 2010". If you have one with a later date, you may
%!      need to make revisions to this document as well. In any event, making
%!      sure your thesis complies with Graduate Education guidelines is
%!      ultimately your responsibility. Caveat LaTeXtor. :)
%!!!!!!!!!!!!!!!!!!!!!!!!!!!!!!!!!!!!!!!!!!!!!!!!!!!!!!!!!!!!!!!!!!!!!!!!!!!!!!

\documentclass[11pt]{uhthesis}

%%% Load some useful packages:
%% New LaTeX2e graphics support
\usepackage{graphicx}
%% Package to linebreak URLs in a sane manner.
\usepackage{url}

%%% Declarations for Front Matter. Capitalize all of these values
%%% "normally". This allows the document class to format them properly.
%% Full title of thesis or dissertation, capitalized like a title should be.
\title{Neutrino Astrophysics with the ANITA3 Telescope}
%% Your name, capitalized normally. Do not include any titles like Dr.
\author{Benjamin J. Rotter}
%% Month in which you intend to receive your degree (i.e. graduation).
%% Presumably this will be one of: May, August, or December.
\degreemonth{April}
%% Year of expected graduation.
\degreeyear{2016}
%% Type of degree to be conferred.
\degree{Doctor of Philosophy}
%% This is the chairperson of your committee. Do not use titles like Dr.
\chair{Peter Gorham}
%% The other members of your committee, seperated by "\\". Again, no titles,
%% and it is customary to list the outside committee member (if you have one)
%% last.
\othermembers{Gary Varner\\
Bob Morse\\
Jelena Maricic\\
Richard Mills
}
%% The field in which you are obtaining your degree, capitalized normally.
\field{Physics}
%% If your discipline allows subfields, you can add it here. Note that this
%% is strictly controlled, so consult the Style & Policy guide before adding
%% a subfield.
%\subfield{Bioinformatics}
%% 4-6 optional keywords/phrases for use in indexing or as search terms
\keywords{theses, dissertations, graduating, misery}
%% The version number of your document. Consistent use of this will enable you
%% to tell old drafts from new ones. Final actual documents omit this
%% automatically so you can use it without fear of submission problems at the
%% end. If you do not define this parameter, it defaults to "1.0.0".
\versionnum{4.0.0}

%%% End of preamble
\begin{document}
\maketitle

\begin{frontmatter}

%%% Note, there is no longer a signature page included in the document, it
%%% has been replaced by Form IV

%%% Create the copyright page (optional)
%\copyrightpage

%%% Bring in the dedication page from external file (optional)
%%%%%%%%%%%%%%%%%%%%%%%%%%%%%%% -*- Mode: Latex -*- %%%%%%%%%%%%%%%%%%%%%%%%%%%%
%% uhtest-dedication.tex -- 
%% Author          : Robert Brewer
%% Created On      : Fri Oct  2 16:29:01 1998
%% Last Modified By: Robert Brewer
%% Last Modified On: Fri Oct  2 16:29:20 1998
%% RCS: $Id: uhtest-dedication.tex,v 1.1 1998/10/06 02:07:25 rbrewer Exp $
%%%%%%%%%%%%%%%%%%%%%%%%%%%%%%%%%%%%%%%%%%%%%%%%%%%%%%%%%%%%%%%%%%%%%%%%%%%%%%%
%%   Copyright (C) 1998 Robert Brewer
%%%%%%%%%%%%%%%%%%%%%%%%%%%%%%%%%%%%%%%%%%%%%%%%%%%%%%%%%%%%%%%%%%%%%%%%%%%%%%%
%% 

\begin{dedication}
\null\vfil
{\large
\begin{center}
To my family,\\\vspace{12pt}
and my friends,\\\vspace{12pt}
I couldn't have done it without you.
\end{center}}
\vfil\null
\end{dedication}


%%% Bring in the acknowledgments section from external file (optional)
%%%%%%%%%%%%%%%%%%%%%%%%%%%%%%% -*- Mode: Latex -*- %%%%%%%%%%%%%%%%%%%%%%%%%%%%
%% uhtest-acknowledgements.tex -- 
%% Author          : Robert Brewer
%% Created On      : Fri Oct  2 16:29:43 1998
%% Last Modified By: Robert Brewer
%% Last Modified On: Fri Oct  2 16:29:52 1998
%% RCS: $Id: uhtest-acknowledgements.tex,v 1.1 1998/10/06 02:06:54 rbrewer Exp $
%%%%%%%%%%%%%%%%%%%%%%%%%%%%%%%%%%%%%%%%%%%%%%%%%%%%%%%%%%%%%%%%%%%%%%%%%%%%%%%
%%   Copyright (C) 1998 Robert Brewer
%%%%%%%%%%%%%%%%%%%%%%%%%%%%%%%%%%%%%%%%%%%%%%%%%%%%%%%%%%%%%%%%%%%%%%%%%%%%%%%
%% 

\begin{acknowledgments}

	Thank you to everyone who helped me and loved me.

\end{acknowledgments}


%%% Bring in the abstract section from external file
%%%%%%%%%%%%%%%%%%%%%%%%%%%%%% -*- Mode: Latex -*- %%%%%%%%%%%%%%%%%%%%%%%%%%%%
%% uhtest-abstract.tex -- 
%% Author          : Ben Rotter
%% Created On      : Fri Oct  2 16:30:18 1998
%% Last Modified By: Ben Rotter
%% Last Modified On: Fri Oct  2 16:30:25 1998
%% RCS: $Id: uhtest-abstract.tex,v 1.1 1998/10/06 02:06:30 rbrewer Exp $
%%%%%%%%%%%%%%%%%%%%%%%%%%%%%%%%%%%%%%%%%%%%%%%%%%%%%%%%%%%%%%%%%%%%%%%%%%%%%%%
%%   Copyright (C) 1998 Robert Brewer
%%%%%%%%%%%%%%%%%%%%%%%%%%%%%%%%%%%%%%%%%%%%%%%%%%%%%%%%%%%%%%%%%%%%%%%%%%%%%%%
%% 

\begin{abstract}
%Basically here I want to talk about the ANITA3 instrument, the theory that is it stuyding (Cosmogenic neutrinos and cosmic rays), and the results of the recent flight.  I should really mention the link between the cosmic rays (even though it seems like the energies we see are lower than the knee cutoff region) and maybe even add in something about darkmatter strangelets in an appendix or something.  I should probably be doing something to link those two things together


%% This is Ben Rotter, and I'm going to write a dissertation.
%This dissertation is an overview of one experimental study of the highest energy particles created in the cosmos, their propogation through space, and their interaction with other media and subsequent detection on Earth. Ultra-high energy cosmic rays (UHECRs), those with energies on the order of $10^{18}eV$, have been observed colliding with matter in the Earth's atmosphere.  These interactions result in the creation of large, energetic particle showers which can be detected by space and ground based detectors.  Due to the extreme energies of these rare particles, much study was poured into determining their cosmic source accelerator.  From this, precision measurements of the flux density over a wide range of energies have been taken, and several interesting features are found within the spectrum.  Most notibly, at the highest observed energy edge of the spectrum, a sudden decrease in flux (or knee) can be observed.  This energy happens to be situated at a resonance for an interaction between a relativistic particle and the low energy photon background radiation, the CMB.  This interaction involves the creation of a delta resonance whose subsequent decay includes a propagating neutrino.  This highly reletivistic neutrino is hypothesized to have high enough flux to be detectable, and has yet to be observed.  The discovery of this cosminogenic neutrino is the main scope of the ANtarctic Impulsive Transient Antenna (ANITA) telescope payload.  This thesis details the theory, experimental design, and analysis of the third flight of the ANITA experiment, culminating in a defendable limit on the flux density of these mysterious particles, as well as a search for UHECRs.

%% Lets try this again
Ultra-high energy cosmic rays (UHECRs), those with energies above $10^{18}$~eV, have been observed colliding with matter in the Earth's atmosphere by a variety of experiments. These particles are amongst the highest observed locally on earth, with energies many orders of magnitude higher than those produced in terrestrial accelerators.  Due to the extreme energies of these rare particles, much study has been poured into determining their cosmic source accelerator, a search which continues today.  This dissertation explores the third flight of the ANtarctic Impulsive Transient Antenna (ANITA) telescope payload in the 2014-2015 Antarctic season, which has a unique opportunity for novel observations of UHE particles.  It covers the cosmological theory, payload overview, detector calibration, signal simulation, and presents results of measurements of UHECRs and Tau Regeneration UHE$\nu$s with the third ANITA flight.

\end{abstract}


%%% Generate table of contents
\tableofcontents

%%% Generate list of tables
%\listoftables

%%% Generate list of figures
%\listoffigures

\end{frontmatter}

%%% Switch to appendix mode
%\appendix

%%% Bring in any appendices from external file (optional)
%%%%%%%%%%%%%%%%%%%%%%%%%%%%%%% -*- Mode: Latex -*- %%%%%%%%%%%%%%%%%%%%%%%%%%%%
%% uhtest-appendix.tex -- 
%% Author          : Robert Brewer
%% Created On      : Fri Oct  2 16:31:12 1998
%% Last Modified By: Robert Brewer
%% Last Modified On: Mon Oct  5 14:41:05 1998
%% RCS: $Id: uhtest-appendix.tex,v 1.1 1998/10/06 02:07:03 rbrewer Exp $
%%%%%%%%%%%%%%%%%%%%%%%%%%%%%%%%%%%%%%%%%%%%%%%%%%%%%%%%%%%%%%%%%%%%%%%%%%%%%%%
%%   Copyright (C) 1998 Robert Brewer
%%%%%%%%%%%%%%%%%%%%%%%%%%%%%%%%%%%%%%%%%%%%%%%%%%%%%%%%%%%%%%%%%%%%%%%%%%%%%%%
%% 

\chapter{Some Ancillary Stuff}

Ancillary material should be put in appendices, which appear before the
bibliography. 

\chapter{More Ancillary Stuff}

Subsequent chapters are labeled with letters of the alphabet.


%% Just for demo purposes, include all entries from bib file
%\nocite{*}

%%% Input file for bibliography
%\bibliography{example}
%% Use this for an alphabetically organized bibliography
%\bibliographystyle{plain}
%% Use this for a reference order organized bibliography
%\bibliographystyle{unsrt}



%%%%%%%%%%%% 1 %%%%%%%%%%%
\chapter{Introduction and Motivation}
%%%%%%%%%%%%%%%%%%%%%%%


\section{Cosmological Theory}
	\subsection{Cosmic Ray Detection History}
	
	\subsection{Cosmic Ray Physics and Cosmology}
	
	\subsection{Unexplained UHE Source Mystery}

	\subsection{GZK interaction}
		
\section{Cosmic Ray Detection on Earth}
	\subsection{Air Showers}
	
	\subsection{Askaryan Effect}

\section{UHE Neutrino detection on earth}
	\subsection{Neutrino hadronic interactions}

	\subsection{Tau-$\nu$ Specific Detection Prospects}


%%%%%%%%%%%%% 2 %%%%%%%%%%%%%
\chapter{ANITA3 Telescope Platform}
%%%%%%%%%%%%%%%%%%%%%%%%%%
\section{Overview}

\section{Quad Ridge Horn Antennas}
	
	\subsection{Antenna Theory}
		
	\subsection{Antenna Response Angle Justification}
	
\section{Analog Filtering}
	\subsection{Importance}
	
	\subsection{Technical Details}
		
	\subsection{Bandwidth Selection Justification} 
		
\section{Amplification}
	\subsection{Expected Signal Power}
	
	\subsection{Dynamic Range}
	
	\subsection{Gain Justification}
	
	\subsection{Noise Figure} 
	
	
	
\section{Digitization}

	\subsection{LABRADOR ASIC}

	\subsection{Limitations}

	\subsection{Impulse Response}
	
	\subsection{Future development (?)}
		
		
\section{Triggering}
	
	\subsection{Trigger Hierarchy Overview}
	
	\subsection{SHORT square-law power integrator}
		
	\subsection{L0 Tiggering efficiency and quality}
		
	\subsection{L0 Optimization}
	
	\subsection{L1 Trigger}
	
	\subsection{L1 Trigger Window Delay Limitations}
		
	\subsection{Phi Sector Masking}
	
\section{Power Monitor}

	\subsection{Sensitivity}

	\subsection{Improvement over ANITA2}

\section{GPS tracking and orientation sensors}

	\subsection{Magnetometer}
	
	\subsection{Sun Sensors}
	
\section{CPU, CPCI, and Data Readout}
	
\section{Data Storage and Telemetry}
	
	\subsection{Redundant data storage}
	
	\subsection{Telemetry}
		
	\subsection{GPU prioritization}
			
%%%%%%%%%%%% 3 %%%%%%%%%%%%%
\chapter{Instrument Calibration}
%%%%%%%%%%%%%%%%%%%%%%%%%%

\section{LABRADOR voltage}

	\subsection{Chip to Chip Variation}
	
	
\section{LABRADOR time}
	\subsection{Time Domain Bin Width Matrix}

	\subsection{Wraparound Time ($\epsilon$)}
	
	\subsection{RCO Phase and determination}
	
	\subsection{Temperature Dependance}
	
	\subsection{dT Variance and Timing Results}
	
	\subsection{Inter-SURF Timing for Waveform Alignment}
	
\section{Antenna location photogrammetry}

\section{Phase center optimization}

	\subsection{WAIS Divide Calibration Pulses}
	
	\subsection{LDB Calibration Pulses}

\section{System Impulse Response}
	\subsection{Effect of signal chain}
	
	\subsection{Convolution with Antenna response}
	
	\subsection{Effect of uneven time sampling}

\section{Triggering Sanity Check}

	\subsection{Angular response characteristics from data}

\section{Non-uniform channels and Outliers}

%%%%%%%%%%%%% 4 %%%%%%%%%%%%%
\chapter{Cosmic Ray Search with ANITA3 Flight Data}
%%%%%%%%%%%%%%%%%%%%%%%%%%
\section{ANITA3 Flight Overview}
	\subsection{In Flight modifications}

\section{Blinding Procedure}

	\subsection{Philosophical Reasoning For Blinding}
	
	\subsection{Threshold for Unblinding Request}

	\subsection{Polarization Specific Blinding}

\section{Event Quality Cuts}

\section{Constant wave source filtering}

	\subsection{Bookkeeping and Filtering Decision Heuristic}

	\subsection{Fourier Domain Band Filtering}
	
	\subsection{Sine Wave Subtraction}

\section{De-dispersion (or maybe just template correlation cut?)}
	\subsection{Wiener Deconvolution}
	
	\subsection{Comparison to simple template correlation cut}

\section{Polarization cuts}
	\subsection{Stokes Parameter Requirements on Signal Events}
	
\section{Event Reconstruction}
	\subsection{Radio Interferometric Pointing}
	
	\subsection{Ray tracing to continent (with index of refraction)}
	
	\subsection{Log-Likelihood two dimensional gaussian pointing error}

	\subsection{Immediate pointing cuts}


\section{Known object categorization}
	\subsection{Sun and its Reflection, Thermal Noise Effect}

	\subsection{Satellites, Bases, and other Anthropogenic Sources}


\section{Thermal noise separation results}
	\subsection{Cut Efficiency}
	
	\subsection{Cut Quality}

\section{Base and non-base clustering}
	\subsection{Clustering Purity and Efficiency}

	

%%%%%%%%%%%%% 5 %%%%%%%%%%%%%
\chapter{Simulation of air shower event in time domain and expected event rate from analysis techniques}
%%%%%%%%%%%%%%%%%%%%%%%%%%
\section{Air Shower Simulation Overview}
	
			
			
%%%%%%%%%%%%% 6 %%%%%%%%%%%%%	
\chapter{Results of Analysis and Limits}
%%%%%%%%%%%%%%%%%%%%%%%%%%%%%
			
			
%%%%%%%%%%%%%%% 7 %%%%%%%%%%%%%%%
\chapter{Quick and Dirty Reanalysis of ANITA2 Data Using Corrected dTs}
%%%%%%%%%%%%%%%%%%%%%%%%%%%%%%%%%%
			
			
			
%%%%%%%%%%%%%% 8 %%%%%%%%%%%%%%%%%%%%		
\chapter{Conclusion and Discussion}
%%%%%%%%%%%%%%%%%%%%%%%%%%%%%%%%%%%%%


%%%%%%%%%%%%%%%%%%%%%%%%%%%		
\chapter{Appendix A: Anti Quark Nuggets}
%%%%%%%%%%%%%%%%%%%%%%%%%%%
\section{Theory Overview}

\section{Possible Interaction Characteristics}

\section{Detection Prospects With ANITA}
	
	
	
\end{document}

