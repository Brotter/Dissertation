%%%%%%%%%%%%%
%
%
% Text for dissertation appendix B and the signal chain response white paper.  
% They will have identical text.
%
% Ben Rotter - University of Hawaii at Manoa - March 2017
%
%%%%%%%%%%%%%%


	The generation of a system impulse response for the ANITA3 instrument from calibration measurements is required to accurately relate the digitized time series of voltages back to an electric field at the instrument.  As the system is sensitive to a wide bandwidth, a full frequency dependent array of phasors is required to completely characterize the system.  The ANITA3 (and more recent ANITA4) flights employ slightly different antennas than the previous ANITA1 and ANITA2 flights in an attempt to capture additional sub 200MHz frequency power. Additionally, the signal chains of the newer flights have subtle modifications preclude direct relation between earlier flights.  This change necessitates a thorough analysis of the full signal chain, from antenna to digitizer.
	
\section{Introduction}

	There are two distinct systems on the ANITA3 instrument that alter the received time domain signal on its to the digitizer, the antennas and the signal chain.  The antennas are broad band horn antennas whose dual polarizations have co-located phase centers.  These antennas couple the propagating electromagnetic (EM) radiation emitted by an extensive comic ray or neutrino air shower (EAS) from into a transmission wire by uniformly shifting the impedance of their impedance from that of free space to that of the RF system.  The collection of amplifiers, filters, and other 50-ohm RF components that populate the path between the antenna and the LABRADOR chip is known as the signal chain, and has it's own frequency dependent gain and phase modulation properties.  To calibrate these two systems, known input signals were injected into the systems and measurements were taken of the output signal.  The methods and techniques used to generate an impulse response using these input and output signals will be discussed, as well as the results for the ANITA3 flight.

\section{Discrete Fourier Transform}

\begin{equation}
\mathcal{F}_{k} = \sum_{n=0}^{N-1} x_{n}e^{-2\pi ikn/N}
\label{eqn:DFT}
\end{equation}

\begin{equation}
x_{n} = \frac{1}{N}\sum_{k=0}^{N-1} \mathcal{F}_{k}e^{-2\pi ikn/N}
\label{eqn:DFT}
\end{equation}

\section{Units and Spectrum Discussion}

	Accurately representing the amount of power in a time domain waveform or network as a function of frequency requires both a normalized Fourier transform that obeys Parseval's Theorem, as well as the appropriate units.  There are two main types of signals which we are considering in this analysis, absolute measurements, referenced to a measured voltage on constant impedance network, and relative measurements relating two different measured signals.  An example of an absolute measurement is the time series waveforms captured on oscilloscopes.  These can be described by decibles referenced to 1mW (dBm) or Power Spectral Density (PSD), represented by dBm/Hz   while the relative measurements are network analyzer S21 measurements and transfer functions.

	The ratio of spectral power between two different signals $x_{A}(t)$ and $x_{B}(t)$, can be done through division of their equivalent Fourier representations $\mathcal{F}_{A}(f)$ and $\mathcal{F}_{B}(f)$.  The logarithmic magnitude of this ratio is what is most often recognized as the gain of the network between the two measured signals.  The calculation of the gain is seen in Equation \ref{eqn:gain}
	
\begin{equation}
Gain(f) = 10log_{10}(|\frac{\mathcal{F}_{A}(f)}{\mathcal{F}_{B}(f)}|^{2}) \qquad [dB]
\label{eqn:gain}
\end{equation}

\begin{equation}
Power(f) = 10log_{10}(\frac{|\mathcal{F}(f)|^{2} * 1000[\frac{mW}{W}]} {Z})\qquad [dBm]
\label{PSD}
\end{equation}
	
\begin{equation}
PSD(f) = 10log_{10}(\frac{|\mathcal{F}(f)|^{2} * 1000[\frac{mW}{W}]} {Z*df})\qquad [dBm/Hz]
\label{PSD}
\end{equation}


\section{Seavey Broad Band Quad-ridge Antenna Impulse Response}

	\subsection{Measurement Goals}

	The ANITA3 instrument utilized 48 broadband antennas 	Many calibration measurements were taken for the antennas used in the ANITA3 flight, each with their own drawbacks and benefits.  The parameters of interest for the antenna measurements are the absolute phase and gain response for any incident EM field, and the directionality and gain pattern as a function of angle off the maximal transmission and receive direction, known as the "boresight" angle.  Additionally the cross polarization fraction is important for determining the full Stokes parameters of a waveform.
	
	\subsection{Measurements Summary}
	
	The three different measurements were taken that are discussed in this paper each probe a unique region of the full calibration measurement.  First, a measurement was taken at the University of Hawaii at Manoa in a copper lined, RF sealed, anechoic chamber using two "identical" (same model and manufacturing batch) ANITA3 flight antennas.  This measurement allowed a maximally noise free environment in which to measure any small signal effects that may be present in the antennas.  It also removes the requirement for correlation and averaging of multiple waveforms, decreasing uncertainty introduced by interpolation techniques.  Unfortunately, the size of the anechoic chamber forced the distance between the antennas into the near field region for the lower side of the frequency bandwidth.  This causes results below 200MHz, with a wavelength of ~1.6m and minimum far field distance requirement of 3.2m, to be slightly distorted and require a 	Fresnel correction. As this is the region in which the antennas were supposed to perform differently than previous ANITA antenna designs, this experimental setup and required correction is highly problematic.  Additionally, measurements of the gain pattern far off the maximal bore-sight angle have Fresnel zones occluded by the absorbing material that lined the walls of the chamber, once again introducing Fresnel interference and increasing the uncertainty in the measurements.
	
	In response to this, additional tests were undertaken at the University of Hawaii with much larger separation distances.  To accomplish this, the two antennas under test were placed on building rooftops separated by a moderately sized courtyard.  The building separation was large enough to escape the near-field of the lower frequency antenna response, yet close enough to preclude any ground bounce interference or multi-path issues.  These measurements were plagued by anthropogenic noise, however a large sample of waveforms uncorrelated to this noise was taken reducing the resulting background significantly.  These measurements provide the best results for the absolute system gain and antenna directivity.
	
	Finally, for the full ANITA flight the consistency and similarity of all 48 antennas needed to be measured.  Bore-sight gain measurements were taken for all antennas in Palestine TX in 2014 during integration of the ANITA3 payload.
	
	\subsection{Absolute Bore-sight Antenna Height}
		
		The relationship between the magnitude of an electric field vector and the induced potential on the transmission line output of the antenna is a combination of the radiation efficiency and beam pattern, and is colloquially known as the effective antenna height, measured in units of volts per meter.  This measurement was done in several  
		
		
	The equation for determining the complex antenna height of a single antenna is given by Equation \ref{eqn:antHeight}.
	
\begin{equation}
\mathcal{H}_{Rx}(f) = \frac{ c r(f) \mathcal{F}_{rec}(f)}{ if \mathcal{F}_{src}(f) \mathcal{H}_{Tx}(f) } 
\label{eqn:antHeight}
\end{equation}

In this equation, $\mathcal{H}(f)$ represents the transfer function of the receive (Rx) and transmit (Tx) antennas, $c$ is the speed of light, $\mathcal{F}(f)$ is the Fourier equivalent sampled waveform, and $r(f)$ is the free space path distance between the phase centers of the antennas.  The location of the phase centers of the antennas are not known, and would require additional absolute timing measurements.  For the purposes of this analysis a linear relationship between frequency and phase center was derived assuming two points, that the 180MHz phase center is at the antenna face, and that the 1.2GHz phase center was at the feed point.

If the antennas are identical, $\mathcal{H}_{Rx}(f) = \mathcal{H}_{Tx}(f)$, and Equation \ref{eqn:antHeight} becomes:

\begin{equation}
\mathcal{H}(f) = \sqrt{ \frac{c r(f)}{if} \frac{\mathcal{F}_{rec}(f)}{\mathcal{F}_{src}(f)} }
\label{eqn:antHeight2}
\end{equation}

Additionally, the complex antenna height relates electric field to measured voltage according to the following equation:

\begin{equation}
\frac{V_{rec}(t)}{\sqrt{Z_{sys}}} = h(t) \circ \frac{ E_{rad}(t) }{ \sqrt{Z_{o}} }
\label{eqn:antHeight2EField}
\end{equation}

Where $\circ$ is the convolution operator, $Z_{sys}$ is the impedance of the transmission line, 50$\Omega$ in our system, and $Z_{o}$ is the impedance of free space, 377$\Omega$.

\section{RF Signal Chain Impulse Response}
		
	\subsection{Measurements Summary}

		
		

\section{Methods and tools}
	
	A variety of methods and tools are required to relate the different calibration data to each other to determine a final transfer function for the system.  Each of these has their own effect on the 

\section{Magnitude Considerations and Absolute Response}
	
