%%%%%%%%%%%%%
%
%
% Text for dissertation appendix B and the signal chain response white paper.  
% They will have identical text.
%
% Ben Rotter - University of Hawaii at Manoa - March 2017
%
%%%%%%%%%%%%%%


	The generation of a system impulse response for the ANITA3 instrument from calibration measurements is required to accurately relate the digitized time series of voltages back to an electric field at the instrument.  As the system is sensitive to a wide bandwidth, a full frequency dependent array of phasors is required to completely characterize the system.  The ANITA3 (and more recent ANITA4) flights employ slightly different antennas than the previous ANITA1 and ANITA2 flights in an attempt to capture additional sub 200MHz frequency power. Additionally, the signal chains of the newer flights have subtle modifications preclude direct relation between earlier flights.  This change necessitates a thorough analysis of the full signal chain, from antenna to digitizer.
	
\section{Introduction}

	There are two distinct systems on the ANITA3 instrument that alter the received time domain signal on its to the digitizer, the antennas and the signal chain.  The antennas are broad band horn antennas whose dual polarizations have co-located phase centers.  These antennas couple the propagating electromagnetic (EM) radiation emitted by an extensive comic ray or neutrino air shower (EAS) from into a transmission wire by uniformly shifting the impedance of their impedance from that of free space to that of the RF system.  The collection of amplifiers, filters, and other 50-ohm RF components that populate the path between the antenna and the LABRADOR chip is known as the signal chain, and has it's own frequency dependent gain and phase modulation properties.  To calibrate these two systems, known input signals were injected into the systems and measurements were taken of the output signal.  The methods and techniques used to generate an impulse response using these input and output signals will be discussed, as well as the results for the ANITA3 flight.


\section{Seavey Broad Band Quad-ridge Antenna Impulse Response}

	\subsection{Measurement Goals}

	The ANITA3 instrument utilized 48 broadband antennas 	Many calibration measurements were taken for the antennas used in the ANITA3 flight, each with their own drawbacks and benefits.  The parameters of interest for the antenna measurements are the absolute phase and gain response for any incident EM field, and the directionality and gain pattern as a function of angle off the maximal transmission and receive direction, known as the "boresight" angle.  Additionally the cross polarization fraction is important for determining the full Stokes parameters of a waveform.
	
	\subsection{Measurements Summary}
	
	The three different measurements were taken that are discussed in this paper each probe a unique region of the full calibration measurement.  First, a measurement was taken at the University of Hawaii at Manoa in a copper lined, RF sealed, anechoic chamber using two "identical" (same model and manufacturing batch) ANITA3 flight antennas.  This measurement allowed a maximally noise free environment in which to measure any small signal effects that may be present in the antennas.  It also removes the requirement for correlation and averaging of multiple waveforms, decreasing uncertainty introduced by interpolation techniques.  Unfortunately, the size of the anechoic chamber forced the distance between the antennas into the near field region for the lower side of the frequency bandwidth.  This causes results below 200MHz, the region in which these new antennas were supposed to perform differently than previous ANITA antenna designs, to be distorted and requires a correction.  Additionally, measurements of the gain pattern far off the maximal bore-sight angle became occluded by the absorbing material that lined the walls of the chamber, worsening the Fresnel interference and increasing the uncertainty in those measurements.
	
	In response to this, additional tests were undertaken at the University of Hawaii with much larger separation distances.  To accomplish this, the two antennas under test were placed on building rooftops separated by a moderately sized courtyard.  The building separation was large enough to escape the near-field of the lower frequency antenna response, yet close enough to preclude any ground bounce interference or multi-path issues.  These measurements were plagued by anthropogenic noise, however a large sample of waveforms uncorrelated to this noise was taken reducing the resulting background significantly.  These measurements provide the best results for the absolute system gain and antenna directivity.
	
	Finally, for the full ANITA flight the consistency and similarity of all 48 antennas needed to be measured.  Bore-sight gain measurements were taken for all antennas in Palestine TX in 2014 during integration of the ANITA3 payload.
	
	\subsection{Absolute Bore-sight Antenna Length}
		
		The relationship between the magnitude of an electric field vector and the induced potential on the transmission line output of the antenna is a combination of the radiation efficiency and beam pattern, and is colloquially known as the effective antenna length, measured in units of volts per meter.  
	
	

\section{RF Signal Chain Impulse Response}


\section{Methods and tools}
	
	A variety 

\section{Magnitude Considerations and Absolute Response}
	
