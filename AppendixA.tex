			
%%%%%%%%%%%%%%%%%%%%%%%%%%%		
\chapter{Anti-Quark Nugget Search with the RF Power Monitor}
%%%%%%%%%%%%%%%%%%%%%%%%%%%
\section{Overview}
	This appendix deals primarily with the detection of Anti-Quark Nuggets, an exotic dark matter candidate particle, using the ANITA power monitor subsystem.
	
	
	\subsection{Sensitivity}
		The sensitivity of the power monitor is driven mainly by the digital averaging of the digitized signal done in firmware.
	\subsection{Improvement over ANITA2}
		The ANITA2 instrument did not have any digital averaging, and relied on a mal-designed analog low pass filter which significantly reduced both the sensitivity of the power monitor and the observed time.  Since the readout was only done at 10Hz, and the integration window was on the order of microseconds, a majority of time was left un-sampled.  This was improved in the ANITA3 flight to include a high factor of digital averaging.  This modification is discussed in detail in Appendix A.
	
