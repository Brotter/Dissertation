			
%%%%%%%%%%%%% 6 %%%%%%%%%%%%%	
\chapter{Results of Analysis and Limits}
%%%%%%%%%%%%%%%%%%%%%%%%%%%%%
\section{Cut Event Results}
	
\section{Event Pointing Results}
	\subsection{Additional Pseudo-Base Candidates}

\section{Background estimate}
	Estimating the number of background events that passed all signal cuts and are present in the final candidate list is import for establishing the confidence in the final result.  This is accomplished using statistical inference about the probability that non-physics events would satisfy all cut requirements~\cite{ClassicalStatisticalEstimation}.  We attempt to divide the events into two sets, one background and one signal, then use the arithmetic of probability to determine the likelihood that an event from the background set could be found within the signal set.  
	
	In order to relate the measured data to uniform 
	
	Bayes' theorem is stated in Equation \ref{eqn:bayesTheorem}, and can be used to determine the estimate of the background.
	\begin{equation}
		P(B | S) = \frac{P(S | B) P(B)}{P(S)}
		\label{eqn:bayesTheorem}
	\end{equation}
	Where P(B) is the number of events in the background set
	
	
	Two different philosophies can be referenced for determining what goes into the background set.  The signal set can be defined as either the expected observation rate of UHECR particles during the flight, or as a single count for each measured candidate event individually which is then summed to find the total background probability.
	
	\subsection{Frequentist Probability}
		%I think Abby's suggested method, simply taking the ratio of impulsive events passing cuts and clustering to all clustered events regardless of impulsivity,  is Frequentist for sure, since it relies only on observed transient producing locations.  
		
	
		
		
	\subsection{Bayesian Probability}
		%However, I think that using the base list to compile distributions where no events passed the impulsivity cuts is more Bayesian, since it includes an assumption that known bases cause impulsive events even if we didn't observe any impulsive events from any specific base.  


\section{Cosmic Ray Candidates}

\section{Cosmic Ray Flux Estimate}


\section{$\nu$-$\tau$ Candidates}

\section{$\nu$-$\tau$ Flux Estimates}