			
%%%%%%%%%%%%% 6 %%%%%%%%%%%%%	
\chapter{Results of Analysis and Limits}
%%%%%%%%%%%%%%%%%%%%%%%%%%%%%
\section{Cut Event Results}
	
\section{Event Pointing Results}
	\subsection{Additional Pseudo-Base Candidates}

\section{Background estimate technique}
	Estimating the number of background events that passed all signal cuts and are present in the final candidate list is accomplished using Bayesian probability.  We attempt to divide the events into two sets, one background, and one signal.  The background events are events that are pointed at known bases and   Bayes' theorem is stated in Equation \ref{eqn:bayesTheorem}.
	\begin{equation}
		P(B | S) = \frac{P(S | B) P(B)}{P(S)}
		\label{eqn:bayesTheorem}
	\end{equation}
	Where P(B) is the number of events in the background set


\section{Cosmic Ray Candidates}

\section{Cosmic Ray Flux Estimate}


\section{$\nu$-$\tau$ Candidates}

\section{$\nu$-$\tau$ Flux Estimates}