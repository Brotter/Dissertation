			
%%%%%%%%%%%%% 6 %%%%%%%%%%%%%	
\chapter{Results of Analysis Cuts, Candidates, Backgrounds, and Limits}
%%%%%%%%%%%%%%%%%%%%%%%%%%%%%


\section{Results of Weak Analysis Cuts}
		After making cuts on the reduced quantities detailed in the previous section, only 5997 events remain from the quality event dataset.  This represents 1.44\times10^{-4} of the initial Hpol triggered dataset.  Table \ref{tab:weakCutReason} describes the number of events that failed each reduced quantity.

	\subsection{Event pointing results}
		Of the 5997 events, when compared against themselves, only 32 fall at least $L>40$ away from any other event in the set.
	
	\subsection{Above horizon events}
		Additionally, 641 pulses in this set occur above the horizon and do not trace back to the continent, but are below zero elevation.  Included in this set would be CRs interacting in the atmosphere being observed directly.
		

	\subsection{Pseudo-base locations}
		The locations where events do cluster can be separated into various pseudo-base locations.  These locations can be seen in Figure \ref{fig:pseudoBases}.

\section{Results of Signal Box Analysis Cuts}
		The final candidate list, which can be determined by making a final signal cut on events within the unclustered ``weak'' event list, results in 16 CR candidate events.  Making these final cuts on the weak event list, regardless of clustering status, yields 116 events.  The 100 events that pass final signal cuts, but cluster with pseudo-bases, are associated with those pseudo-bases in Table \ref{tab:pseudoBaseCandidates}.
		
	\subsection{Event pointing results}
		The locations of the final 16 candidates can 

\section{Background Estimates}
	Estimating the number of background events that passed all signal cuts and are present in the final candidate list is import for establishing the confidence in the final result.  The strong cut on the 

\section{Diffuse Source Frequentist Background Estimate}
	 This is accomplished using statistical inference about the probability that non-physics events would satisfy all cut requirements~\cite{ClassicalStatisticalEstimation}.  We attempt to divide the events into two sets, one background and one signal, then use the arithmetic of probability to determine the likelihood that an event from the background set could be found within the signal set.  

	In order to relate the measured data to uniform 
	
	Bayes' theorem is stated in Equation \ref{eqn:bayesTheorem}, and can be used to determine the estimate of the background.

	\begin{equation}
		P(B | S) = \frac{P(S | B) P(B)}{P(S)}
	\label{eqn:bayesTheorem}
	\end{equation}
	
	Where P(B) is the number of events in the background set
	
	
	Two different philosophies can be referenced for determining what goes into the background set.  The signal set can be defined as either the expected observation rate of UHECR particles during the flight, or as a single count for each measured candidate event individually which is then summed to find the total background probability.
	
	\subsection{Frequentist Probability}
		%I think Abby's suggested method, simply taking the ratio of impulsive events passing cuts and clustering to all clustered events regardless of impulsivity,  is Frequentist for sure, since it relies only on observed transient producing locations.  
		
		
\section{Point Source Frequentist Background Estimate}
	
\section{ABCD Background Estimate}
		The ABCD method is a common particle physics background estimator in which the ratio of background event populations within two areas of parameter space, combined with the number of events with characteristics placing it in a third, neighboring, space, can be used to derive the expected background count in a signal box region.  This expected background can then be compared versus the number of events that are actually seen to occur within that signal box.  Assuming that the distribution is constant throughout the four regions for background, and that signal only lies within the signal box, any upward deviation from the expected background count can be assumed to be the signal.
		
		This method presents issues 
		


\section{Cosmic Ray Candidates}

	\subsection{Reflected events}
	
	\subsection{Direct events}
	
	\subsection{Energy Estimates}


\section{Cosmic Ray Flux Estimate}

	\subsection{Livetime}

	\subsection{Instrumented Volume}
	
	\subsection{Cut Efficiency}
	

\section{$\nu_{\tau}$ Candidate Unblinding}
	Up to this point, the candidates have been presented without reference to their polarity.  After setting the cuts, determining the background estimate, and compiling a list of events, permission to look at the results of the polarity estimator was asked of the collaboration.  After unblinding the data, it was determined that the two above horizon events had the same polarity, and were 


	
%\section{$\nu_{\tau}$ Flux Estimates}




%	\subsection{Bayesian Probability}
		%However, I think that using the base list to compile distributions where no events passed the impulsivity cuts is more Bayesian, since it includes an assumption that known bases cause impulsive events even if we didn't observe any impulsive events from any specific base.  