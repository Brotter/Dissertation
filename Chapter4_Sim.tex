%%%%%%%%%% 4 %%%%%%%%%%%%%%%%
\chapter{Simulation of Air Shower Event in Time Domain and Expected Event Rate from Analysis Techniques}
%%%%%%%%%%%%%%%%%%%%%%%%%%

%
%
%
%   I'm going to integrate this into the analysis section
%
%
%



\section{Air Shower Simulation Overview}
	The electromagnetic field emmission from cosmic ray air showers has been extensively modeled using the ZHAires package.  This package tracks the charge distribution as the shower develops, calculates the induced radiation from the current density as a function of time and location from the shower core, and propagates the radiation to the ice and its reflection back to the payload.  
	
	\subsection{ZHAires Shower Modeling}
		Simulating the radio emission of cosmic ray air showers is accomplished by using the ZHS extension to the AIRshower Extended Simulations (AIRES) package.\cite{AlvarezMuñiz2012325}  Concatenated, these two packages make up the ZHAires simulation framework.  This framework, developed by Jaime Alvarez-Muniz, Washington Rodrigues de Carvalho Jr. and Enrique Zas, generates a time varient electric field for a configurable incident particle at some instrumentation point in the atmosphere.  Recent work, with ANITA specifically in mind, has further extended the simulation to accurately depict reflections off the ice sheet.
		
		Simulations provide an important method to determine the overall sensitivity of the ANITA instrument to air shower events as a function of incident angle, energy, and payload location.  Due to the coherent beamed characteristic of air shower radio emmission, specifically a peak at the critical Cherenkov angle from the shower axis, ANITA is only sensitive to a small fraction of showers that occur within its field of view.
		
		The spectral content of EAS radiation falls off sharply as a function of offset from the critical Cherenkov angle.  Without being within a few degrees of the peak coherence angle, low frequency radiation, below the band of the ANITA instrument, dominates the spectrum.  This can be seen in Figure figure.  
		
\section{Convolution of Modeled Shower and System Response}
		
		\subsection{System Impulse Response}
		
		\subsection{System Noise Addition}
		

\section{IceMC: ANITA Monte Carlo Simulation Package}
	An additional simulation monte-carlo package named "icemc" utilizes these electromagnetic shower simulations and allows a comparison between that both generates a realistic incident particle flux, as well as models the trigger and instrument response of ANITA, 
\section{Instrumented Volume Estimate}